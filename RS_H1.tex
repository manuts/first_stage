\section{Construction of the parity check matrix using Reed-Solomon code}\label{chap:"RS"}

The parity check matrix of a LDPC code is a sparse matrix. Ideally the Bipartite graph constructed using the  parity check matrix should not contain a 
cycle (no feedback) for optimal decoding. However, in practice it is hard to generate parity check matrices without cycles. Nevertheless most well 
defined methods for generating LDPC codes try to avoid cycles of length less than some minimum length (
minimum length could be any of the even positive integer like $6$, $8$, $10$, etc.). In this chapter we will discuss LDPC code construction based on 
Reed-Solomon codes\cite{IEEE1}, also known as RS-LDPC codes. The bipartite graph for these codes does not have cycle of length 4. 
A $(6,32)$ regular $(2048,1723)$ RS-LDPC code has been adopted as the FEC in the IEEE $802.3$an $10$GBase-T standard\cite{IEEE}. 

\subsection{Reed-Solomon codes} 

Consider the Galois field $GF(q)$ with $q$ elements, where $q$ is a positive integer power of a prime number. Let $\rho$ be a positive integer 
such that $2 \leq \rho $<$ q$. The generator polynomial of cyclic $(n,k,d_{min})$ code $C$ is given by:
\begin{align}
\ g(X)&= (X-\alpha)(X-\alpha^2)\cdot\cdot\cdot(X-\alpha^{\rho-2})\\
\ &= g_0+g_1X+\cdot\cdot\cdot+X^{\rho-2}
\end{align} 

Notice that $n=q-1$, $k=q-\rho+1$, $g_i\in GF(q)$ and $\alpha$ is a primitive element of a field. The parity check matrix $R_H$ for a 
Reed-Solomon code has size  $(\rho -2) \times n$. Any submatrix of size $(\rho -2)\times(\rho -2)$ of parity check matrix is the Vandermonde matrix. 
Therefore linear combination of any $(\rho -2)$ column will not result in $0_v$. The rank of matrix $R_H$ can be utmost $(\rho -2)$. Thus minimum distance is $d_{min}=(\rho -1)$. 

Now consider the $(q-1)$-tuple vector $g^{(0)}=(g_0,g_1,...,g_{\rho-2},0,0,...,0)$. Notice that $g_{\rho-2}=1$. By cyclically shifting $g^{(0)}$, we get generator 
matrix $G$ of size $k\times n$ for code $C$.
\[ G= \left( \begin{array}{cccccccccc}
g_0 & g_1 & g_2 & . & . & 1 & 0 & . & . & 0\\
0 & g_0 & g_1 & g_2 & . & . & 1 & .0 & . & . \\
\vdots & \vdots & \vdots & \vdots & \vdots & \vdots & \vdots & \vdots & \vdots & \vdots\\
0 & . & . & g_0 & g_1 & g_2 & . & . & . & 1 \end{array} \right)\] 

The generator matrix for shortened RS code $C_b$ is a submatrix of size $2 \times \rho$ and it is shown below:
\begin{align*}
 G_b= \begin{pmatrix}
g_0 & g_1 & g_2 & . & . & . & 1 & 0\\
0 & g_0 & g_1 & g_2 & . & . & . &1 \end{pmatrix}
\end{align*} 

\subsection{Properties of $C_b$} 
\begin{enumerate}
\item Since the length of codewords in $C_b$ is $\rho$ and the minimum distance between two codewords of $C_b$ is $(\rho-1)$, two codewords in $C_b$ only agree at most at one location.
\item Let $c$ be a codeword of weight $\rho$. If we multiply $c$ by $\forall \beta \in GF(q)$, we get set $C_b^{(1)}$ of $(q-1)$ codewords of weight $\rho$. 
    Now, length of every codeword of $C_b^{(1)}$ is also $\rho$. So $C_b^{(1)}$ is a MDS (Maximum Distance Separable) code.
\item Let us partition $C_b$ into a $q$ cosets $C_b^{(1)},C_b^{(2)},...,C_b^{(q)}$ based on $C_b^{(1)}$. Notice that $C_b^{(i)}$ is MDS code. 
    Therefore two codewords in any coset $C_b^{(i)}$ must differ in all the locations. 
\end{enumerate}

\subsection{Construction}
Let us now explain the explicit construction procedure for a LDPC code check matrix $H$.
\begin{enumerate}
\item All $q$ elements of $GF(q)$ can be expressed as some power of a primitive element $\alpha$. Let us define the location vector of $\alpha^i$ as a $q$-tuple over $GF(2)$, $$z(\alpha^i)=(0,0,...,1,0,...,0)$$, where $i^{th}$ element of $z(\alpha^i)$ is $1$ and all other elements are $0$. Choose one codeword $b=(b_1,b_2,...,b_{\rho}) \in C_b^{(i)}$. If we replace each $b_i$ ($1 \leq i \leq \rho $) by its location vector $z(b_i)$, we get a $z(b)=(z(b_1),z(b_2),...,z(b_{\rho}))$, which is a $ \rho q$-tuple of weight $\rho$ over $GF(2)$.
\item Arrange all $q$ $\rho q$-tuple of $C_b^{(i)}$ as rows of a matrix and call this matrix as $A_i$. The weight of each column of $A_i$ is $1$. 
\item Choose a positive integer $\gamma$, such that $1 \leq \gamma \leq q$. Then the parity check matrix $H$ of size $\gamma q \times \rho q$ is defined as:
\begin{align}
 H \overset{\triangle}{=}  \begin{bmatrix}
A_1\\
A_2\\
\vdots\\
A_{\gamma} \end{bmatrix} 
\end{align}
\end{enumerate}
Since each column of $A_i$ has weight $1$, weight of an each column of $H$ is $\gamma$. So, $H$ is a $(\gamma , \rho)$-regular matrix. Each row in $A_i$ is a coset member, so each row in $A_i$ is different,
\begin{enumerate}
\item Two rows in $A_i$ do not have single common element, 
\item Two codewords in $C_b$ agree at at most one symbol location ($d_{min}=\rho-1$). 
\end{enumerate}
From Property~$1$ and $2$ above, no two rows from $A_i,~A_j,~i\neq j$ agree at more than a single element. This will imply that the bipartite graph corresponding to $H$ is free of length 4 cycles. To see this, notice that our construction guarantees a $H$ which contains no rectangle having all the corner points non-zero. 

In this chapter we have shown the construction of a ($\gamma,\rho$) regular RS-LDPC. This code has girth at least six. In the next chapter we show the construction of an irregular LDPC code.